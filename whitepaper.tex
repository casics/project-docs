\documentclass{casicswhitepaper}

\begin{document}
\title{White paper}
\date{2016-01-11}
\maketitle

\section{Introduction}

We discuss the results of a survey we conducted in September, 2015. \note{We need a lot more here.}

\note{Links to similar work.}

\section{Survey structure}

The survey was designed to capture current practices and experiences in searching for software from two distinct groups: those looking for ready-to-run software and those looking for source code to incorporate into their own code base respectively. It consisted of 37 questions subdivided into these areas of concern. Questions in the first section aimed to establish the relative importance of different search criteria. Those in the second section characterized the experiences of the developer. Two optional open-ended response questions were also asked, requesting specific case histories and feedback on the survey.

\section{Survey demographics}

The survey was advertised on mailing lists and social media forums oriented to the astronomical and biological sciences and particularly to computational subcommunities within those domains. Of the 68 respondents, 54\% identified as working in the physical sciences, 44\% in computing and maths, 28\% in biological sciences and 12\% in others (multiple areas of work were allowed). Assuming a typical 8 hour working day, 94\% regularly spent more than four hours of their time engaged with software and 68\% more than six hours. 88\% of respondents also said that they had were free to make software choice decisions. 

These point to a balanced response to the survey from the targeted scientific communities by computer literate individuals. Amongst the 81\% who subsequently indicated that they were involved in software development to some degree (and not just end users), the median number of years of software development experience was 20. This also suggests that the typical respondent is mid-career or part of the (earlier) home computer generation where programming rather than gaming was more common. This may also indicate a possible bias in responses against more junior members of the respective communities, such as students and postdocs, who may have different search criteria and development experiences than their more experienced colleagues. This should be borne in mind when interpreting the results. It may also be the result of a degree of self selection, in that 
more knowledgeable individuals are more likely to participate in community surveys and something that we should aim to redress in future similar efforts.

\section{Survey results}

\subsection{Software search criteria}

Personal recommendations and general search engines were the two main ways in which survey respondees 

\subsection{Developer experiences}

\end{document}